\documentclass{article} 

\usepackage{exercise}
\usepackage{amsmath}

\usepackage[utf8]{inputenc}
\usepackage[swedish]{babel}

\title{Matematik 5 - Matematik 5000, kapitel 1}
\author{Vilhelm Prytz}
\date{September 2020}

\begin{document}
\maketitle

\setcounter{Exercise}{1191}

\begin{Exercise}
    a)
    \[
        1 \ 9 \ 36 \ 84 \ 126 \ 126 \ 84 \ 36 \ 9 \ 1
    \]
    b)
    \[
        1 \ 10 \ 45 \ 120 \ 210 \ 252 \ 210 \ 120 \ 45 \ 10 \ 1
    \]

    \[
        \binom{10}{0} \ \binom{10}{1} \ \binom{10}{2} \ \binom{10}{3} \ \binom{10}{4} \ \binom{10}{5} \
        \binom{10}{6} \ \binom{10}{7} \ \binom{10}{8} \ \binom{10}{9} \ \binom{10}{10} \
    \]

    Vi söker termen där koefficienten är 120. Enligt bionomialsatsen bör det vara \(a^{n-k}b^k\).
    Eftersom raden i pascals triangel kan beskrivas med \(\binom{n}{k}\) är termerna vi söker \(120a^7b^3\) och \(120a^3b^7\).

\end{Exercise}

\setcounter{Exercise}{1195}
\begin{Exercise}

    \[
        \binom{12}{4} = \binom{11}{4} + \binom{11}{3}
    \]

    Exempel: En person tänker arbeta 4 månader nästa år. Visa att summan av antalet urval där juli ingår och antalet där juli inte ingår är lika med antalet sätt som 4 månader av 12 kan väljas ut. \newline

    Förklaring:

    \( \binom{12}{4} \) betyder antalet sätt att välja 4 månader av 12.

    Vi bestämmer att en månad av de månader som ska ingå blande de 4 "låses fast". T.ex. låses juli fast.

    Antalet sätt att kombinera de 4 månader blir då \(1 \cdot \binom{11}{3}\) (eftersom en av månaderna nu är fastlåst).

    Om juli inte ingår, så ska jag bestämma antal sätt att välja 4 bland 11.

    Båda fallen kompletterar varandra. Alltså blir \( \binom{12}{4} = \binom{11}{4} + \binom{11}{3} \).

\end{Exercise}

\begin{Exercise}

    Visa att

    \[
        \binom{k}{k} + \binom{k+1}{k} + ... + \binom{n}{k} = \binom{n+1}{k+1}
    \]

    Vi börjar med HL och använder Pascals formel, \(\binom{n}{k} = \binom{n-1}{k}+\binom{n-1}{k-1}\).

    \[
        HL = \binom{n+1}{k+1} = \binom{(n+1)-1}{k+1} + \binom{(n+1)-1}{(k+1)-1}
    \]
    \[
        = \binom{n}{k} + \binom{n}{k+1}      
    \]
    Nu fortsätter vi att använda Pascals formel på den sista.
    \[
        = \binom{n}{k} + \binom{n-1}{k} + \binom{n-1}{k+1}
    \]
    \[
        = \binom{n}{k} + \binom{n-1}{k} + \binom{n-2}{k} + \binom{n-2}{k-1}
    \]
    \[....\]
    \[
        = \binom{n}{k} + \binom{n-1}{k} + \binom{n-2}{k} + \ ... \ + \binom{k+2}{k} + \binom{k+2}{k+1} 
    \]
    Till slut kommer vi hamna till detta.
    \[
        = \binom{n}{k} + \binom{n-1}{k} + \binom{n-2}{k} + \ ... \ + \binom{k+2}{k} + \binom{k+1}{k} + \binom{k+1}{k+1} = HL
    \]

    Eftersom \(\binom{k}{k} = 1\) och \(\binom{k+1}{k+1} = 1\) är alltså \(VL = HL\), V.S.V.

\end{Exercise}
\end{document}