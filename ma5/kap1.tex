\documentclass{article} 

\usepackage{exercise}
\usepackage{amsmath}

\usepackage[utf8]{inputenc}
\usepackage[swedish]{babel}

\title{Matematik 5 - Matematik 5000, kapitel 1}
\author{Vilhelm Prytz}
\date{September 2020}

\begin{document}
\maketitle

\setcounter{Exercise}{1195}
\begin{Exercise}

    \[
        \binom{12}{4} = \binom{11}{4} + \binom{11}{3}
    \]

    \( \binom{12}{4} \) betyder antalet sätt att välja 4 månader av 12.

    Vi bestämmer att en månad av de månader som ska ingå blande de 4 "låses fast". T.ex. låses juli fast.

    Antalet sätt att kombinera de 4 månader blir då \(1 \cdot \binom{11}{3}\) (eftersom en av månaderna nu är fastlåst).

    Om juli inte ingår, så ska jag bestämma antal sätt att välja 4 bland 11.

    Båda fallen kompletterar varandra. Alltså blir \( \binom{12}{4} = \binom{11}{4} + \binom{11}{3} \).

\end{Exercise}

\begin{Exercise}

    Visa att

    \[
        \binom{k}{k} + \binom{k+1}{k} + ... + \binom{n}{k} = \binom{n+1}{k+1}
    \]

\end{Exercise}


\end{document}